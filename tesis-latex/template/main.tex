%-------------------------------------------------------------------------------
%	PREÁMBULO
%-------------------------------------------------------------------------------

\documentclass[11pt, oneside]{book}
\usepackage[paperwidth=17cm, paperheight=22.5cm, bottom=2.5cm, right=2.5cm]{geometry}

% El borde inferior puede parecerles muy amplio a la vista. Les recomiendo hacer una prueba de impresión antes para ajustarlo

\usepackage{amssymb,amsmath,amsthm} % Símbolos matemáticos
\usepackage[spanish]{babel}
\usepackage[utf8]{inputenc} % Acentos y otros símbolos 
\usepackage{enumerate}
\usepackage{hyperref} % Hipervínculos en el índice
\usepackage{graphicx}
%\usepackage{subfig} % Subfiguras
\graphicspath{{Imagenes/}} % En qué carpeta están las imágenes

% Para eliminar guiones y justificar texto
\tolerance=1
\emergencystretch=\maxdimen
\hyphenpenalty=10000
\hbadness=10000

\linespread{1.25} % Asemeja el interlineado 1.5 de Word

\let\oldfootnote\footnote % Deja espacio entre el número del pie de página y el inicio del texto
\renewcommand\footnote[1]{%
\oldfootnote{\hspace{0.05mm}#1}}

\renewcommand{\thefootnote} {\textcolor{Black}{\arabic{footnote}}} % Súperindice a color negro

\setlength{\footnotesep}{0.75\baselineskip} % Espaciado entre notas al pie

\usepackage{fnpos} % Footnotes al final de pág.

\usepackage[justification=centering, font=bf, labelsep=period, skip=5pt]{caption} % Centrar captions de tablas y ponerlas en negritas

\newcommand{\imagesource}[1]{{\footnotesize Fuente: #1}}

\usepackage{tabularx} % Big tables
\usepackage{graphicx}
\usepackage{adjustbox}
\usepackage{longtable}

\usepackage{float} % Float tables

\usepackage[usenames,dvipsnames]{xcolor} % Color

\usepackage{pgfplots} % Gráficas
\pgfplotsset{compat=newest}
\pgfplotsset{width=7.5cm}
\pgfkeys{/pgf/number format/1000 sep={}}

\begin{document}

%-------------------------------------------------------------------------------
%	PORTADA
%-------------------------------------------------------------------------------

\title{Titulo del Trabajo} % Con este nombre se guardará el proyecto en writeLaTex

\begin{titlepage}
\begin{center}

\textsc{\Large Instituto Tecnológico Autónomo de México}\\[2em]

%Figura
\begin{figure}[h]
\begin{center}
\includegraphics[scale=0.50]{itam_logo.png}
\end{center}
\end{figure}

% Pueden modificar el tamaño del logo cambiando la escala

\textbf{\LARGE Titulo del Trabajo}\\[2em]

\textsc{\large Tesis}\\[1em] % Tesis, Tesina, o Caso

\textsc{\large que para obtener el título de}\\[1em]

\textsc{\LARGE Maestra en Ciencias de Datos}\\[1em]

\textsc{\large Presenta}\\[1em]

\textsc{\LARGE Nombre estudiante}\\[1em]

\textsc{\large Asesora}\\[1em]

\textsc{\LARGE Dra. }\\[2em]

% Asegúrense de escribir el nombre completo de su asesor

\end{center}

\vspace*{\fill}
\textsc{Ciudad de México \hspace*{\fill} 2018}

\end{titlepage}

%-------------------------------------------------------------------------------
%	DECLARACIÓN
%-------------------------------------------------------------------------------

\thispagestyle{empty}

\vspace*{\fill}
\begingroup

\noindent
«Con fundamento en los artículos 21 y 27 de la Ley Federal del Derecho de Autor y como titular de los derechos moral y patrimonial de la obra titulada ``\textbf{Título del Trabajo}'', otorgo de manera gratuita y permanente al Instituto Tecnológico Autónomo de México y a la Biblioteca Raúl Bailléres Jr., la autorización para que fijen la obra en cualquier medio, incluido el electrónico, y la divulguen entre sus usuarios, profesores, estudiantes o terceras personas, sin que pueda percibir por tal divulgación una contraprestación.»

% Asegúrense de cambiar el título de su tesis en el párrafo anterior

\centering 

\vspace{5em}

\rule[1em]{20em}{0.5pt} % Línea para la fecha

\textsc{Fecha}
 
\vspace{8em}

\rule[1em]{20em}{0.5pt} % Línea para la firma

\textsc{Farid Hannan Goyri}

\endgroup
\vspace*{\fill}

%-------------------------------------------------------------------------------
%	DEDICATORIA
%-------------------------------------------------------------------------------

\pagestyle{plain}
\frontmatter

\chapter*{}
\begin{flushright}
\textit{Linea 1,\\ Línea 2.}
\end{flushright}

%-------------------------------------------------------------------------------
%	AGRADECIMIENTOS
%-------------------------------------------------------------------------------

\chapter*{Agradecimientos}

\noindent Lorem ipsum dolor sit amet, consectetur adipiscing elit.

% Esta sección es lo único que la gente lee. True story :)

%-------------------------------------------------------------------------------
%	RESUMEN
%-------------------------------------------------------------------------------

\chapter*{Resumen}

\noindent 

\pagestyle{plain}

\noindent 

%-------------------------------------------------------------------------------
%	Summary
%-------------------------------------------------------------------------------

\chapter*{Summary}

\noindent 

\pagestyle{plain}

\noindent 

%-------------------------------------------------------------------------------
%	TABLA DE CONTENIDOS
%-------------------------------------------------------------------------------

\tableofcontents

%-------------------------------------------------------------------------------
%	ÍNDICE DE CUADROS Y FIGURAS
%-------------------------------------------------------------------------------

\listoftables

\listoffigures

%-------------------------------------------------------------------------------
%	TESIS
%-------------------------------------------------------------------------------

\mainmatter % Empieza la numeración de las páginas

\pagestyle{plain}

% Incluye los capítulos en el fólder de capítulos

\chapter*{Introducción}
\addcontentsline{toc}{chapter}{Introducción}

% La introducción no cuenta como primer capítulo

\noindent La Economía tiene distintas definiciones. 

% Se sugiere que el primer párrafo de cada sección no tenga sangría: \noindent


\chapter{Revisión de literatura}

\noindent Este capítulo está dividido en dos secciones.  

\newpage

\section{Modelos teóricos}

\subsection{Hechos estilizados}


\chapter{Título de segundo capítulo}

\section{Introudcción}

Esto es un ejemplo de cita textual.

\begin{quote}
    \small{Mientras que la distinción entre los cerebros de niños y niñas empieza biológicamente, estudios recientes muestran que es \textit{solo} el comienzo. La estructura cerebral no está escrita sobre piedra en el nacimiento ni al final de la infancia, como antes se creía, sino que continúa cambiando a lo largo de la vida. Más que ser inmutable, nuestros cerebros son mucho más plásticos y cambiables de lo que los científicos creían hace una década. El cerebro humano es también la máquina de aprendizaje más talentosa que conocemos. Así que nuestra cultura y el cómo nos enseñaron a comportarnos desempeñan un papel importante en el diseño y reestructura de nuestros cerebros (Brizendine 2010, 5-6).}
\end{quote}

Esto es un ejemplo de tabla. 

\begin{table}[H]
\centering
\caption{Índices de modernidad y tradicionalismo}
\label{PHEL}
\begin{tabular}{|ccc|}
\hline
 País & Índice de modernidad & Índice de tradicionalismo  \\ 
\hline
Alemania & 0.58 & 0.45 \\
Austria & 0.55 & 0.49  \\
Bélgica & 0.50 & 0.49  \\
Canadá & 0.61 & 0.50 \\
Dinamarca & 0.58 & 0.44 \\ 
España & 0.47 & 0.62 \\
Estados Unidos & 0.59 & 0.44  \\
Finlandia & 0.62 & 0.38 \\
Francia & 0.49 & 0.59 \\ 
Holanda & 0.58 & 0.49 \\ 
Irlanda & 0.54 & 0.59  \\
Islandia & 0.63 & 0.54 \\
Italia & 0.56 & 0.58  \\
Japón & 0.42 & 0.48 \\
Noruega & 0.53 & 0.44 \\
Portugal & 0.50 & 0.71 \\ 
Reino Unido & 0.56 & 0.54  \\
Suecia & 0.62 & 0.51 \\
Promedio & 0.58 & 0.51 \\
\hline
\end{tabular}

\begin{tabular}{c}
\footnotesize{Fuente: Bojilov y Phelps (2012).}
\end{tabular}

\end{table}

% Para diseñar tablas


\include{Chapters/3.Mod}

\include{Chapters/4.Mex}

\chapter*{Conclusiones}
\addcontentsline{toc}{chapter}{Conclusiones}

% No enumerar el capitulo de conclusiones. 



%-------------------------------------------------------------------------------
%	APÉNDICES
%-------------------------------------------------------------------------------

\begin{appendix}

\include{Apendices/ApA}

\end{appendix}

%-------------------------------------------------------------------------------
%	BIBLIOGRAFÍA
%-------------------------------------------------------------------------------


\chapter*{Referencias}
\addcontentsline{toc}{chapter}{Referencias}

% Macro. Esto es muy importante, no lo borren

\makeatletter
\renewenvironment{thebibliography}[1]
     {\@mkboth{\MakeUppercase\refname}{\MakeUppercase\refname}%
      \list{}%
           {\setlength{\labelwidth}{0pt}%
            \setlength{\labelsep}{0pt}%
            \setlength{\leftmargin}{\parindent}%
            \setlength{\itemindent}{-\parindent}%
            \@openbib@code
            \usecounter{enumiv}}%
      \sloppy
      \clubpenalty4000
      \@clubpenalty \clubpenalty
      \widowpenalty4000%
      \sfcode`\.\@m}
     {\def\@noitemerr
       {\@latex@warning{Empty `thebibliography' environment}}%
      \endlist}
\makeatother

\begin{thebibliography}{111}

% Lista

% La manera recomendada para citar papers o libros en el formato de Chicago esta en el siguiente vínculo: https://www.chicagomanualofstyle.org/tools_citationguide/citation-guide-2.html

% Es importante poner el apellido del autor seguido del año de publicación, una coma y las páginas consultadas en el texto antes de puntuar y entre paréntesis para las citas en el cuerpo de la tesis

% Ejemplo:

% Las \textit{causas próximas} del crecimiento son conocidas: tecnología, capital humano y físico. La pregunta es ¿por qué unos países sí tienen estas causas próximas y otros no? La respuesta son las \textit{causas fundamentales:} suerte, geografía, cultura e instituciones (Acemoglu 2009, 110).

%AAAAA
\bibitem{abram57} Abramovitz, Moses. 1957. «Resources on Output Trends in the United States since 1870.» \textit{The American Economic Review} 46 (2): 5–23.

\bibitem{acemoglu09} Acemoglu, Daron. 2009. \textit{Introduction to Modern Economic Growth.} Princeton: Princeton University Press.

%BBBBB

%CCCCC

%DDDDD

%EEEEE

%FFFFF

%GGGGG

%HHHHH

%IIIII

%JJJJJ

%KKKKK

%LLLLL

%MMMMM

%NNNNN

%OOOOO

%PPPPP

%QQQQQ

%RRRRR

%SSSSS

%TTTTT

%UUUUU

%VVVVV

%WWWWW

%XXXXX

%YYYYY

%ZZZZZ

\end{thebibliography}

\end{document}